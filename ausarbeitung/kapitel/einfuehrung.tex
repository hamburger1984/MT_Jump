%---------------------------------------------------------------------------------------------------
% Einführung
%---------------------------------------------------------------------------------------------------
\newpage

\section{Einführung}

Die Raumfahrt ist der Versuch der Beherrschung komplexer Technologie.
Dennoch geschehen Unfälle, die gerade in der Start- und Landephase für Gerät und Astronauten katastrophal enden.
Die letzten Unglücke mit umgekommenen Personen waren die Space Shuttles Columbia im Jahre 2003 und Challenger im Jahre 1986.
Im Falle der Challenger gibt es Stimmen, die behaupten, dass die Crew mit einem Rettungssystem hätte überleben können.
Ein Rettungszenario wäre Ausstieg, freier Fall bis in dichtere Atmosphärenschichten und Landung mittels Fallschirm.
Gesteigertes Interesse an der Sicherheit gibt es angesichts der zunehmenden privaten Raumfahrt und der Vision Privatpersonen als Touristen in den Orbit zu transportieren.

Hierzu hat der Getränkehersteller Red Bull im Jahre 2012 medienwirksam ein Experiment finanziert.
Der Österreicher Felix Baumgartner stieg am 14.10.2012 in einer Kapsel mittels Heliumballon in eine Höhe von $39.045m$ auf (geplant waren mindestens $35.576m$) und sprang.
Neben einem großen Spektakel und neuen Rekorden wurde damit der Beweis erbracht, dass der Sprung aus derartigen Höhen möglich und dieser Weg als Rettungstrategie denkbar ist.

Der Fallschirmsprung aus der Stratosphäre wurde z.B. im Rahmen des Projekts \emph{Excelsior}~\cite{af.mil:excelsior} bereits in den Jahren 1959 und 1960 getestet.
Hierbei lag der Fokus allerdings mehr auf Rettungstrategien für hoch fliegende Jets.
Grundlegend wurde hierbei bereits die Machbarkeit von Fallschirmsprüngen aus einer Höhe von $30km$ bewiesen.
Der damalige Springer Joseph Kittinger führte drei Sprünge durch.
Es gibt umstrittene Behauptungen, dass beim letzten Sprung aus $31.333m$ Höhe die Schallmauer durchbrochen wurde.
Seit 2009 ist Kittinger am Red Bull Projekt beteiligt und half so beratend beim Brechen seiner eigenen Rekorde mit.

Um die Frage zu klären, aus welcher Höhe mindestens gesprungen werden muss, um die Schallmauer zu durchbrechen soll im Folgenden das Esperiment simuliert werden.
Dabei soll untersucht werden, wie der Sprung aus unterschiedlichen Höhen verlaufen wäre und welche Faktoren dabei eine Rolle spielen.

% Das Überschreiten vermeintlicher Grenzen gepaart mit Neugier (und die Aussicht auf Ruhm) sind ein starker Antrieb des Menschen.
% Luft- und Raumfahrt sind eine Frucht dieser Motive.
% Trotz aller Vorkehrungen werden hierbei Risiken eingegangen.
% Im Falle der Raumfahrt zwar nur von einigen wenigen Menschen, trotzdem gibt es Bestrebungen hier die Sicherheit zu erhöhen.
% Red Bull hat hierzu medienwirksam ein Experiment finanziert:
% Der Sprung des österreichischen Felix Baumgartners aus einer geplanten Höhe von 36.576m.
% Neben der Werbewirkung und dem Aufstellen neuer Rekorde sollten dabei medizinische Daten gesammelt und die Machbarkeit eines Notausstiegs von Astronauten in großer Höhe und der Rückkehr im freien Fall getestet werden.
