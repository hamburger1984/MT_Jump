%---------------------------------------------------------------------------------------------------
% Einführung
%---------------------------------------------------------------------------------------------------
\newpage

\section{Einführung}

Die Raumfahrt ist der Versuch der Beherrschung komplexer Technologie.
Dennoch geschehen Unfälle, die gerade in der Start- und Landephase für Gerät und Astronauten katastrophal enden.
Die letzten Unglücke mit umgekommenen Personen waren die Space Shuttles Columbia im Jahre 2003 und Challenger im Jahre 1986.
Im Falle der Challenger gibt es Stimmen, die behaupten, dass die Crew mit einem Rettungssystem hätte überleben können.
Ein Rettungszenario wäre Ausstieg, freier Fall bis in dichtere Atmosphärenschichten und Landung mittels Fallschirm.

Hierzu hat der Getränkehersteller Red Bull im Jahre 2012 medienwirksam ein Experiment finanziert.
Der Österreicher Felix Baumgartner stieg am 14.10.2012 in einer Kapsel mittels Heliumballon in eine Höhe von $39.045m$ auf (geplant waren mindestens $35.576m$) und sprang.
Neben einem großen Spektakel und neuen Rekorden wurde damit der Beweis erbracht, dass der Sprung aus derartigen Höhen möglich ist.

Im Folgenden soll das Experiment simuliert und untersucht werden, wie der Sprung aus anderen Höhen verlaufen wäre und welche Faktoren dabei eine Rolle spielen.

% Das Überschreiten vermeintlicher Grenzen gepaart mit Neugier (und die Aussicht auf Ruhm) sind ein starker Antrieb des Menschen.
% Luft- und Raumfahrt sind eine Frucht dieser Motive.
% Trotz aller Vorkehrungen werden hierbei Risiken eingegangen.
% Im Falle der Raumfahrt zwar nur von einigen wenigen Menschen, trotzdem gibt es Bestrebungen hier die Sicherheit zu erhöhen.
% Red Bull hat hierzu medienwirksam ein Experiment finanziert:
% Der Sprung des österreichischen Felix Baumgartners aus einer geplanten Höhe von 36.576m.
% Neben der Werbewirkung und dem Aufstellen neuer Rekorde sollten dabei medizinische Daten gesammelt und die Machbarkeit eines Notausstiegs von Astronauten in großer Höhe und der Rückkehr im freien Fall getestet werden.
