%---------------------------------------------------------------------------------------------------
% Schluss
%---------------------------------------------------------------------------------------------------
\section{Zusammenfassung}\label{sec:schluss}

Es wurde gezeigt, dass das Durchbrechen der Schallmauer im freien Fall unter den angenommenen Voraussetzungen erst ab einer Absprunghöhe von $35240m$ möglich ist.
Dieser Wert ist mit einer gewissen Unsicherheit zu verstehen, da weder von Red Bull noch von anderer Seite eindeutige sichere Angaben für den $c_w$-Wert und die Anströmfläche $A$ zu finden waren.
Weiterhin ist zu berücksichtigen, dass die getroffenen Aussagen für das gewählte Atmosphärenmodell für einen Ort und einen Zeitpunkt gelten.
Die Durchführung eines derartigen Sprungs bei veränderter Wetterlage oder an anderer Stelle auf dem Globus kann zu anderen Ergebnissen führen.

Interessant ist die Erkenntnis, dass sich unterschiedliche Absprunghöhen nur bis zu einer Höhe von ca. $15000m$ auswirken.
Ab dieser Höhe unterscheiden sich die Geschwindigkeitsverläufe nur noch minimal.

\newpage
