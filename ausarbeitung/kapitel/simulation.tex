%---------------------------------------------------------------------------------------------------
% Simulation
%---------------------------------------------------------------------------------------------------
\section{Simulation}\label{sec:simulation}

Die Simulation wurde mit Hilfe von Matlab und Simulink durchgeführt.
Sie verfügt über die bereits besprochenen Attribute $c_w$-Wert, Anströmfläche $A$ und die Masse $m_S$. Weiterhin kann eine Absprunggeschwindigkeit $v_0$ und Abprunghöhe $h_0$ angegeben werden.

Die Funktion für $a_S$ (Formel~\ref{f:a_s}) wurde als Matlabfunktion (Listing~\ref{m:beschleunigung}) umgesetzt.
Eine weitere Matlabfunktion (Listing~\ref{m:c}) berechnet die Schallgeschwindigkeit in Abhängigkeit von der aktuellen Temperatur in Kelvin.

\lstinputlisting[caption={Matlabfunktion Beschleunigung},label=m:beschleunigung]{includes/beschleunigung.m}

\lstinputlisting[caption={Matlabfunktion Schallgeschwindigkeit},label=m:c]{includes/c.m}

Bei Erreichen der Höhe $0m$ wird die Simulation gestoppt.

Die Werte für Höhe, Geschwindigkeit, Beschleunigung und Schallgeschwindigkeit werden zur Analyse und Ausgabe in die aktuelle Workspace abgelegt.

\begin{figure}[h]
  \centering
  \includegraphics[width=1\textwidth]{simulink}
  \caption{Aufbau der Simulation}
  \label{fig:simulink}
\end{figure}

Die Simulation wurde für verschiedene Absprunghöhen durchgeführt.
\begin{itemize}
  \item Zur Kalibierung wurde zunächst der Sprung mit $h_0=39045$ durchgeführt (siehe Abb.~\ref{fig:39045}).
  \item Zur Ermittlung der benötigten Mindesthöhe zum Durchbrechen der Schallmauer wurde die Simulation mittels Skript in Schritten á $2000m$ für  $24000m$ bis $40000m$ Höhe durchgeführt.
\end{itemize}

\begin{figure}[h]
  \centering
  \includegraphics[width=1\textwidth]{speed_vs_height_baumgartner}
  \caption{Geschwindigkeit und Schallgeschwindigkeit, Sprung aus 39045m Höhe}
  \label{fig:39045}
\end{figure}

Resultierende Geschwindigkeiten und die in der Höhe geltende Schallgeschwindigkeit wurden gegen die Höhe in einem Graph geplottet.

% \subsection{Fallschirm}
% Eine kleine Anpassung der Simulation setzt bei Erreichen der Höhe $1500m$ den $c_w$-Wert auf $1.4$ und die Anströmfläche $A$ auf $25$.
% Dies soll das Öffnen eines Fallschirms simulieren.
